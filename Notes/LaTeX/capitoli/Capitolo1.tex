\chapter{Introduzione}

\section{Contesto e Motivazione}

\begin{figure}[htbp]
    \centering
    \includegraphics[width=0.7\textwidth]{immagini/placeholder-figure.png}
    \caption{Esempio di figura con didascalia.}
    \label{fig:esempio-figura}
\end{figure}

In questa sezione viene introdotto il contesto generale e la motivazione del lavoro. Il riferimento a una figura può essere fatto così: Figura~\ref{fig:esempio-figura}.

\subsection{Stato dell'Arte}
Lo stato dell'arte rappresenta una panoramica sulle conoscenze attuali e sugli sviluppi recenti nel campo oggetto di studio. È possibile evidenziare punti di interesse utilizzando \textcolor{unipd-red}{colori} o \textbf{grassetto}.

\subsubsection{Sottosezione con Formule}
È possibile inserire formule matematiche nel testo $E = mc^2$ o in display:

\begin{equation}
    f(x) = \int_{-\infty}^{\infty} \hat{f}(\xi) e^{2\pi i \xi x} d\xi
    \label{eq:esempio-formula}
\end{equation}

La formula~\ref{eq:esempio-formula} mostra un esempio di trasformata di Fourier inversa.

\paragraph{Paragrafo con Elenco}
È possibile creare elenchi numerati o puntati:

\begin{itemize}
    \item Primo elemento dell'elenco puntato
    \item Secondo elemento con \textit{testo in corsivo}
    \item Terzo elemento con riferimento all'equazione~\ref{eq:esempio-formula}
\end{itemize}

\subparagraph{Sottoparagrafo con Tabella}
Le tabelle possono essere formattate in modo professionale:

\begin{table}[htbp]
    \centering
    \caption{Esempio di tabella.}
    \label{tab:esempio-tabella}
    \begin{tabular}{lcr}
        \toprule
        \textbf{Intestazione 1} & \textbf{Intestazione 2} & \textbf{Intestazione 3} \\
        \midrule
        Valore 1 & Valore 2 & Valore 3 \\
        Valore 4 & Valore 5 & Valore 6 \\
        \bottomrule
    \end{tabular}
\end{table}

\section{Esempio di Codice}
È possibile includere blocchi di codice formattati:

\begin{lstlisting}[language=Python, caption=Esempio di codice Python]
def calcola_fibonacci(n):
    """Calcola l'n-esimo numero di Fibonacci."""
    if n <= 1:
        return n
    else:
        return calcola_fibonacci(n-1) + calcola_fibonacci(n-2)
        
# Test della funzione
for i in range(10):
    print(f"Fibonacci({i}) = {calcola_fibonacci(i)}")
\end{lstlisting}

\section{Citazioni e Riferimenti}
È possibile utilizzare citazioni nel testo \cite{riferimento1} e creare una bibliografia completa a fine documento. Per citazioni più estese:

\begin{quote}
    ``Le citazioni estese possono essere formattate in questo modo, utilizzando l'ambiente quote. È possibile aggiungere l'autore della citazione a fine paragrafo.''
    \begin{flushright}
        --- Autore della Citazione
    \end{flushright}
\end{quote}

\epigraph{È anche possibile utilizzare l'ambiente epigraph per inserire citazioni all'inizio di capitoli o sezioni.}{--- Autore dell'Epigrafe}

\section{Note e Conclusioni}
Questa sezione può contenere note conclusive o riassuntive del capitolo. È possibile anche inserire note a piè di pagina\footnote{Esempio di nota a piè di pagina.}.